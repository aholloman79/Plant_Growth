% Options for packages loaded elsewhere
\PassOptionsToPackage{unicode}{hyperref}
\PassOptionsToPackage{hyphens}{url}
%
\documentclass[
]{article}
\usepackage{amsmath,amssymb}
\usepackage{iftex}
\ifPDFTeX
  \usepackage[T1]{fontenc}
  \usepackage[utf8]{inputenc}
  \usepackage{textcomp} % provide euro and other symbols
\else % if luatex or xetex
  \usepackage{unicode-math} % this also loads fontspec
  \defaultfontfeatures{Scale=MatchLowercase}
  \defaultfontfeatures[\rmfamily]{Ligatures=TeX,Scale=1}
\fi
\usepackage{lmodern}
\ifPDFTeX\else
  % xetex/luatex font selection
\fi
% Use upquote if available, for straight quotes in verbatim environments
\IfFileExists{upquote.sty}{\usepackage{upquote}}{}
\IfFileExists{microtype.sty}{% use microtype if available
  \usepackage[]{microtype}
  \UseMicrotypeSet[protrusion]{basicmath} % disable protrusion for tt fonts
}{}
\makeatletter
\@ifundefined{KOMAClassName}{% if non-KOMA class
  \IfFileExists{parskip.sty}{%
    \usepackage{parskip}
  }{% else
    \setlength{\parindent}{0pt}
    \setlength{\parskip}{6pt plus 2pt minus 1pt}}
}{% if KOMA class
  \KOMAoptions{parskip=half}}
\makeatother
\usepackage{xcolor}
\usepackage[margin=1in]{geometry}
\usepackage{color}
\usepackage{fancyvrb}
\newcommand{\VerbBar}{|}
\newcommand{\VERB}{\Verb[commandchars=\\\{\}]}
\DefineVerbatimEnvironment{Highlighting}{Verbatim}{commandchars=\\\{\}}
% Add ',fontsize=\small' for more characters per line
\usepackage{framed}
\definecolor{shadecolor}{RGB}{248,248,248}
\newenvironment{Shaded}{\begin{snugshade}}{\end{snugshade}}
\newcommand{\AlertTok}[1]{\textcolor[rgb]{0.94,0.16,0.16}{#1}}
\newcommand{\AnnotationTok}[1]{\textcolor[rgb]{0.56,0.35,0.01}{\textbf{\textit{#1}}}}
\newcommand{\AttributeTok}[1]{\textcolor[rgb]{0.13,0.29,0.53}{#1}}
\newcommand{\BaseNTok}[1]{\textcolor[rgb]{0.00,0.00,0.81}{#1}}
\newcommand{\BuiltInTok}[1]{#1}
\newcommand{\CharTok}[1]{\textcolor[rgb]{0.31,0.60,0.02}{#1}}
\newcommand{\CommentTok}[1]{\textcolor[rgb]{0.56,0.35,0.01}{\textit{#1}}}
\newcommand{\CommentVarTok}[1]{\textcolor[rgb]{0.56,0.35,0.01}{\textbf{\textit{#1}}}}
\newcommand{\ConstantTok}[1]{\textcolor[rgb]{0.56,0.35,0.01}{#1}}
\newcommand{\ControlFlowTok}[1]{\textcolor[rgb]{0.13,0.29,0.53}{\textbf{#1}}}
\newcommand{\DataTypeTok}[1]{\textcolor[rgb]{0.13,0.29,0.53}{#1}}
\newcommand{\DecValTok}[1]{\textcolor[rgb]{0.00,0.00,0.81}{#1}}
\newcommand{\DocumentationTok}[1]{\textcolor[rgb]{0.56,0.35,0.01}{\textbf{\textit{#1}}}}
\newcommand{\ErrorTok}[1]{\textcolor[rgb]{0.64,0.00,0.00}{\textbf{#1}}}
\newcommand{\ExtensionTok}[1]{#1}
\newcommand{\FloatTok}[1]{\textcolor[rgb]{0.00,0.00,0.81}{#1}}
\newcommand{\FunctionTok}[1]{\textcolor[rgb]{0.13,0.29,0.53}{\textbf{#1}}}
\newcommand{\ImportTok}[1]{#1}
\newcommand{\InformationTok}[1]{\textcolor[rgb]{0.56,0.35,0.01}{\textbf{\textit{#1}}}}
\newcommand{\KeywordTok}[1]{\textcolor[rgb]{0.13,0.29,0.53}{\textbf{#1}}}
\newcommand{\NormalTok}[1]{#1}
\newcommand{\OperatorTok}[1]{\textcolor[rgb]{0.81,0.36,0.00}{\textbf{#1}}}
\newcommand{\OtherTok}[1]{\textcolor[rgb]{0.56,0.35,0.01}{#1}}
\newcommand{\PreprocessorTok}[1]{\textcolor[rgb]{0.56,0.35,0.01}{\textit{#1}}}
\newcommand{\RegionMarkerTok}[1]{#1}
\newcommand{\SpecialCharTok}[1]{\textcolor[rgb]{0.81,0.36,0.00}{\textbf{#1}}}
\newcommand{\SpecialStringTok}[1]{\textcolor[rgb]{0.31,0.60,0.02}{#1}}
\newcommand{\StringTok}[1]{\textcolor[rgb]{0.31,0.60,0.02}{#1}}
\newcommand{\VariableTok}[1]{\textcolor[rgb]{0.00,0.00,0.00}{#1}}
\newcommand{\VerbatimStringTok}[1]{\textcolor[rgb]{0.31,0.60,0.02}{#1}}
\newcommand{\WarningTok}[1]{\textcolor[rgb]{0.56,0.35,0.01}{\textbf{\textit{#1}}}}
\usepackage{graphicx}
\makeatletter
\def\maxwidth{\ifdim\Gin@nat@width>\linewidth\linewidth\else\Gin@nat@width\fi}
\def\maxheight{\ifdim\Gin@nat@height>\textheight\textheight\else\Gin@nat@height\fi}
\makeatother
% Scale images if necessary, so that they will not overflow the page
% margins by default, and it is still possible to overwrite the defaults
% using explicit options in \includegraphics[width, height, ...]{}
\setkeys{Gin}{width=\maxwidth,height=\maxheight,keepaspectratio}
% Set default figure placement to htbp
\makeatletter
\def\fps@figure{htbp}
\makeatother
\setlength{\emergencystretch}{3em} % prevent overfull lines
\providecommand{\tightlist}{%
  \setlength{\itemsep}{0pt}\setlength{\parskip}{0pt}}
\setcounter{secnumdepth}{-\maxdimen} % remove section numbering
\ifLuaTeX
  \usepackage{selnolig}  % disable illegal ligatures
\fi
\usepackage{bookmark}
\IfFileExists{xurl.sty}{\usepackage{xurl}}{} % add URL line breaks if available
\urlstyle{same}
\hypersetup{
  pdftitle={Plant Growth},
  pdfauthor={Avery Holloman},
  hidelinks,
  pdfcreator={LaTeX via pandoc}}

\title{Plant Growth}
\author{Avery Holloman}
\date{2024-11-02}

\begin{document}
\maketitle

\begin{Shaded}
\begin{Highlighting}[]
\FunctionTok{library}\NormalTok{(readxl)}
\NormalTok{Plant\_Growth\_500\_Rows }\OtherTok{\textless{}{-}} \FunctionTok{read\_excel}\NormalTok{(}\StringTok{"C:/Users/jacob/Downloads/Plant\_Growth\_500\_Rows.xlsx"}\NormalTok{)}
\FunctionTok{View}\NormalTok{(Plant\_Growth\_500\_Rows)}
\end{Highlighting}
\end{Shaded}

\begin{Shaded}
\begin{Highlighting}[]
\CommentTok{\# Since I have already instealled "readxl" package, it is time to load library}
\FunctionTok{library}\NormalTok{(readxl)}

\CommentTok{\# Lets take a closer look into the structure of the dataset}
\FunctionTok{str}\NormalTok{(Plant\_Growth\_500\_Rows)}
\end{Highlighting}
\end{Shaded}

\begin{verbatim}
## tibble [500 x 2] (S3: tbl_df/tbl/data.frame)
##  $ weight: num [1:500] 5.5 4.86 5.65 6.52 4.77 ...
##  $ group : chr [1:500] "treatment1" "treatment1" "treatment2" "treatment2" ...
\end{verbatim}

\begin{Shaded}
\begin{Highlighting}[]
\CommentTok{\# ANOVA would be usefull to analyze the weight differences across all of the groups}
\NormalTok{anova\_results }\OtherTok{\textless{}{-}} \FunctionTok{anova}\NormalTok{(}\FunctionTok{lm}\NormalTok{(weight }\SpecialCharTok{\textasciitilde{}}\NormalTok{ group, }\AttributeTok{data =}\NormalTok{ Plant\_Growth\_500\_Rows))}
\FunctionTok{print}\NormalTok{(anova\_results)}
\end{Highlighting}
\end{Shaded}

\begin{verbatim}
## Analysis of Variance Table
## 
## Response: weight
##            Df Sum Sq Mean Sq F value Pr(>F)
## group       2   2.34 1.17065  1.2169  0.297
## Residuals 497 478.12 0.96202
\end{verbatim}

\begin{Shaded}
\begin{Highlighting}[]
\CommentTok{\# To different the differences I will add colors to help specifically assign the colors to each group}
\NormalTok{colors }\OtherTok{\textless{}{-}} \FunctionTok{c}\NormalTok{(}\StringTok{"lightblue"}\NormalTok{, }\StringTok{"lightgreen"}\NormalTok{, }\StringTok{"salmon"}\NormalTok{)}

\CommentTok{\# Boxplot}
\FunctionTok{boxplot}\NormalTok{(weight }\SpecialCharTok{\textasciitilde{}}\NormalTok{ group, }\AttributeTok{data =}\NormalTok{ Plant\_Growth\_500\_Rows, }
        \AttributeTok{ylab =} \StringTok{"Weight"}\NormalTok{, }\AttributeTok{main =} \StringTok{"Plant Growth by Group"}\NormalTok{,}
        \AttributeTok{col =}\NormalTok{ colors, }\AttributeTok{border =} \StringTok{"darkgray"}\NormalTok{)}

\CommentTok{\# Adding a legend to represent the colors for each group}
\FunctionTok{legend}\NormalTok{(}\StringTok{"topright"}\NormalTok{, }\AttributeTok{legend =} \FunctionTok{levels}\NormalTok{(}\FunctionTok{as.factor}\NormalTok{(Plant\_Growth\_500\_Rows}\SpecialCharTok{$}\NormalTok{group)),}
       \AttributeTok{fill =}\NormalTok{ colors, }\AttributeTok{border =} \StringTok{"darkgray"}\NormalTok{, }\AttributeTok{title =} \StringTok{"Group"}\NormalTok{)}
\end{Highlighting}
\end{Shaded}

\includegraphics{Plant-Growth-in-R-Studio_files/figure-latex/unnamed-chunk-2-1.pdf}

\end{document}
